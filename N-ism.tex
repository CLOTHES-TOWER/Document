\documentclass[conference]{IEEEtran}
\IEEEoverridecommandlockouts
% The preceding line is only needed to identify funding in the first footnote. If that is unneeded, please comment it out.
\usepackage{cite}
\usepackage{amsmath,amssymb,amsfonts}
\usepackage{algorithmic}
\usepackage{graphicx}
\usepackage{textcomp}
\usepackage{xcolor}
\def\BibTeX{{\rm B\kern-.05em{\sc i\kern-.025em b}\kern-.08em
    T\kern-.1667em\lower.7ex\hbox{E}\kern-.125emX}}
\begin{document}

\title{Clothes Tower : Smart Closet\\
{\footnotesize \textsuperscript{*}N-ism}
\thanks{Identify applicable funding agency here. If none, delete this.}
}

\author{\IEEEauthorblockN{JunSung Kim}
\IEEEauthorblockA{\textit{dept. of Information System} \\
\textit{Hanyang University}\\
Seoul, Republic of Korea \\
jsistop16@naver.com}
\and
\IEEEauthorblockN{SeungHwan Cheon}
\IEEEauthorblockA{\textit{dept. of Information System} \\
\textit{Hanyang University}\\
Seoul, Republic of Korea \\
dg7989@hanyang.ac.kr}
\and
\IEEEauthorblockN{JeMin Seo}
\IEEEauthorblockA{\textit{dept. of Information System} \\
\textit{Hanyang University}\\
Seoul, Republic of Korea \\
jemin3161@naver.com}
\and
\IEEEauthorblockN{PyeongSoo Park}
\IEEEauthorblockA{\textit{dept. of Information System} \\
\textit{Hanyang University}\\
Seoul, Republic of Korea \\
ps3624@naver.com}
}

\maketitle

\begin{abstract}
People have a lot of clothes. Did you forget anything? Are there clothes that you can't find when you need them? Clothes Tower is a smart closet that allows users to use clothes more conveniently by receiving the  type and unique number of clothes through the application. Users will be able to wear the clothes they want in the best condition at any time through the Clothes Tower.
\end{abstract}

\begin{IEEEkeywords}
Clothes, Closet, management, Application, Furniture
\end{IEEEkeywords}


\begin{table}
\caption{Role Assignments}
\label{t1}
\begin{tabular}{|p{2cm}|p{2cm}|p{3cm}|}
\noalign{\smallskip}\noalign{\smallskip}\hline
Roles & Name & Task descriptions and etc \\
\hline
User & PyeongSoo Park & It analyzes and investigates which services customers want. It analyzes the purpose of the closet in detail and investigates how customers feel satisfied with the closet. It is in charge of various fields such as design and function.  \\
\hline
Customer & JeMin Seo & It is in charge of the overall planning of ideas. From the customer's point of view, it contemplates and presents which parts are inconvenient and should be added. As the project progresses, problems are first discovered.  \\
\hline
Software Developer  & JunSung Kim & Create data that drives the closet and construct a draft on how to induce it to work. Design and implement applications. Overall, it is responsible for solving the driving method and software problems of the item.  \\
\hline
Development Manager  & SeungHwan Cheon & Identify and lead the overall flow of planning. Check the direction of the plan from time to time to see if it meets the original intention. In addition, it manages whether the tasks of the previous roles are being performed properly.  \\
\hline
\end{tabular}
\end{table}

\section{Introduction}

\subsection{Motivation / Problem Statement (client’s needs)}

Recently, people's interest in fashion is getting hotter. Previously, there were many people who valued only expensive brand clothing. However, these days, each person buys and wears clothes according to their own personality. Wear many kinds of clothes such as outerwear, top, bottom, shoes, and accessories. In addition, as styles diversify, people share their own methods through SNS (Instagram or YouTube). Therefore, consumers sometimes forget the clothes they have organized in their closets because they repeatedly wear, buy, and throw away various clothes. Even once the season changes, there may be cases where you don't know which clothes were there. In addition, there will not be many people who manage many clothes with only one closet. We thought about how to manage many clothes in one space. And we found the answer at the parking tower where the vehicle was stored.
The parking tower is a mixture of elevators and parking lots. The purpose of the parking tower is to maximize space utilization with the aim of parking many vehicles in a narrow space. For example, it is commonly used in high-rise buildings with many people due to the high floor of hotels and buildings where the circulation of cars is not fast. In addition, the advantage of the parking tower is that it is easy to monitor parking conditions, easy to operate equipment, and easy to operate and maintain as messages are printed in the event of a failure.
From now on, we will compare and analyze the 'Clothes Tower' and the parking tower to find client’s needs.

\begin{enumerate}
    \item Space Utilization
    \begin{itemize}
        \item Parking Tower
        \begin{description}
            \item As the number of vehicles to be accommodated increases, it is difficult to solve with the existing parking method (using the entire floor as a parking space). This is because it is far from design and utilization to divide one floor into parking and office spaces. Therefore, if you use it as a parking space, you have no choice but to use all floors as a parking space. Therefore, it was used vertically and long to increase the utilization of the building.
        \end{description}
        \item Clothes Tower
        \begin{description}
            \item As clothes became more diverse and more diverse, there was a problem of lack of storage space. In addition, since clothes worn in different seasons vary depending on the season, clothes worn in different seasons often forget where they were placed when they were taken out and worn. Therefore, Cloth Tower, like the parking tower, adopted a horizontal extension method for use in the house. In the case of closets, it was judged that the entire space could be used, such as built-in cabinets and dress rooms.
        \end{description}
    \end{itemize}
    \item Easy to Manipulate
    \begin{itemize}
        \item Parking Tower
        \begin{description}
            \item The car is stocked and shipped through the screen. When the vehicle is put in at the time of warehousing and the number of the vehicle is entered, the machine remembers the information and places the vehicle in an empty space in the parking tower. When shipped, the vehicle is moved like an elevator to the position of the borrower with only a simple license plate input operation. Therefore, it is possible to relieve the hassle of driving separately or finding a vehicle.
        \end{description}
        \item Clothes Tower
        \begin{description}
            \item Clothes can be taken out and put in through the display shown outside the closet. When putting clothes in, simply enter and store information about clothes. And store clothes in an empty space. When taking out clothes, the closet aligns corresponding clothes according to the style and needs desired by the user through previous data. This, like the parking tower, can solve the hassle of users wandering around looking for clothes.
        \end{description}
    \end{itemize}
        \item Operation and Maintenance
    \begin{itemize}
        \item Parking Tower
        \begin{description}
            \item If there is a machine failure or error through the screen, it is delivered through message output. Therefore, the machine operator only needs to check the facility without having to check everywhere, so time and cost are saved.
        \end{description}
        \item Clothes Tower
        \begin{description}
            \item In case the closet fails to properly show the clothes the user wants, the application linkage method was chosen. Since clothes can be managed and maintained through applications as well as displays in the closet, users can prepare clothes in advance regardless of location. Therefore, it guarantees time saving for the user.
        \end{description}
    \end{itemize}
\end{enumerate}

\subsection{Research on any related software}
\begin{itemize}
    \item $Acloset$
    \begin{description}
        \item ‘Acloset’ is an application that manages clothes that users have. There are four categories in total. It consists of a home screen, a shopping screen, a registered clothes management screen, and a style management screen. The home screen shows an analysis of the weather, recommended styles, clothes worn this week, and styles. On the shopping screen, products that customers need are sold by classifying them by type of clothes. The screen for managing registered clothes shows the current status of clothes that users have by adding and deleting clothes themselves. The style management screen shows how the user will dress in advance by adding style.
    \end{description}
        \item $OTTOK$
    \begin{description}
        \item ‘OTTOK’ is an application that analyzes the style when the user registers the clothes and matches the coordination. When registering clothes, you can register them separately by type. In addition, you can register by item or by coordination when registering. When a user coordinates clothes registered, the AI of the application analyzes them. Based on this information, it is possible to compare or watch coordination styles with other users.
    \end{description}
            \item $Amazon Echo Look$
    \begin{description}
        \item ‘Amazon Echo Look’ is an AI and camera that advises users on what clothes look good on them through machine learning after performing a 360-degree 3D scan. It is equipped with AI, so you can ask about the weather or schedule. Photos or videos taken through a camera can be checked through a smartphone, saved, or shared on SNS. Also, sending photos to AI recommends clothes that suit users better through machine learning.
    \end{description}
\end{itemize}


\section{REQUIREMENT}

\subsection{Software}
\begin{enumerate}
    \item Hanger type page
    \begin{itemize}
        \item Adding/Deleting clothes
        \begin{description}
            \item Basically, clothes are added one by one. Enter the basic information of the clothes and transmit the data of the clothes to the server when adding them. The added clothes are searched as a list within the corresponding page. When deleting, you can delete the clothes in the list by clicking.
        \end{description}
        \item Searching for clothes
        \begin{description}
            \item User can inquire about clothes in the list and search for specific clothes. User enters a search word based on the basic information of the clothes, user will see the clothes that meet the conditions.
        \end{description}
        \item Ventilation system
        \begin{description}
            \item A ventilator in a hanger-type closet can be operated.
        \end{description}
        \item Styler
        \begin{description}
            \item If user enters the desired date and time among the registered clothes, the styler function is activated.
        \end{description}
    \end{itemize}
    \item Drawer type page
    \begin{description}
        \item It is a page that can adjust the environment of each drawer. It is a drawer that can hold similar types of clothes, and it is possible to maintain an optimal condition for each drawer by adjusting temperature and humidity. It controls access to the desired drawer.
    \end{description}
    \item Calendar and Weather page
    \begin{description}
        \item The user's schedule may be registered. It helps users easily check schedules and actively utilize styler functions. In addition, the weather API helps you choose what to wear.
    \end{description}
\end{enumerate}

\subsection{Hardware}
\begin{enumerate}
    \item Hanger type space
    \begin{itemize}
        \item Closet
        \begin{description}
            \item It is largely divided into an entrance part of clothes used as a styler and a storage space for clothes. The storage space is separated for each clothing. The access of clothes is managed through the rail.
        \end{description}
        \item Styler
        \begin{description}
            \item It is located at the entrance to the clothes. It has the same function as the styler on the market and manages the smell and contamination of clothes.
        \end{description}
        \item Partition
        \begin{description}
            \item It prevents stains due to the color of the clothes through the partition. In addition, when partitions are installed, static electricity is prevented, and lint and dust pollution that occur when clothes overlap each other is prevented in advance.
        \end{description}
        \item Ventilation fan
        \begin{description}
            \item It is responsible for the overall ventilation of the closet-type space. Air circulation in the closet-type space prevents the unique quaint smell of the closet.
        \end{description}
    \end{itemize}
    \item Drawer type space
    \begin{description}
        \item All drawers move through rails. It controls access to drawers using applications.
    \end{description}
    \item Display
    \begin{description}
        \item The application and function are the same, and user can manage closet yourself without a cell phone.
    \end{description}
\end{enumerate}

\subsection{Software and Hardware Interaction}
\begin{enumerate}
    \item When selecting the clothes/drawers listed on the hanger/drawer type closet page of the application, clothes come out through the rail to the exit.
    \item When executing the styler function on the hanger-type closet page of the application, run the function on the styler on the entrance side of the hanger-type closet according to the date and time desired by the user.
    \item When executing the ventilation system on the hanger-type closet page of the application, run the ventilator in the hanger-type closet.
    \item Adjust the temperature and humidity of the drawer when setting the temperature and humidity on the drawer-type closet page of the application.
    \item The functions of the closet display and application interact the same as hardware.
\end{enumerate}


\section{Development Environment}

\subsection{Choice of software development platform}
\begin{itemize}
    \item Which platform and Why?
    \begin{enumerate}
        \item Android
        \begin{figure}[htbp]
        \centerline{\includegraphics[width=3cm]{Android.png}}
        \caption{Android}
        \label{fig}
        \end{figure}
        \begin{description}
            \item Our service chose to develop an app instead of a website because users have to manage their clothes remotely without being affected by the location. The next options were IOS, android, and cross-platform applications. There are team members who have experienced Android development, and we chose the Android app to make the most of the native features of the Android operating system.
        \end{description}
        \item Ubuntu Linux environment of AWS EC2
        \begin{figure}[htbp]
        \centerline{\includegraphics[width=3cm]{AWS.png}}
        \caption{AWS EC2}
        \label{fig}
        \end{figure}
        \begin{description}
            \item The server operates in the Ubuntu Linux environment of AWS EC2. Since AWS’ Free tier service is used, the service can be distributed quickly without any cost burden. In this project, we plan to actively use AWS services. DB used AWS RDS. Since the free tier includes the use of MYSQL DB, it was possible to build a stable DB without cost. DB can be easily managed remotely in conjunction with MYSQL Workbench 8.0 CE of the local PC.
        \end{description}
        \item Node.js
        \begin{figure}[htbp]
        \centerline{\includegraphics[width=3cm]{nestjs_노드js 프레임워크입니당.png}}
        \caption{Node.js}
        \label{fig}
        \end{figure}
        \begin{description}
            \item Node.js was selected as the server-side development language. Node.js is a JavaScript runtime of a single thread asynchronous model that can be learned relatively easily by people unfamiliar with server programming. In addition, data in JSON format can be easily processed. Due to the nature of the project, we chose node.js because we do not use data analysis or machine learning, and there are not many CPU-intensive tasks. ( Express, a representative backend framework of node.js, is light and convenient, but it is difficult to experience structured programming because developers have to devise the structure of the program from scratch. There is also a disadvantage in that it is difficult to use Swagger, an API documentation tool to be used for this project. So, in this project, we decided to use a node.js framework called Nest.js, which is similar to JAVA’s Spring Framework. Nest.js judged that the framework would help improve backend capabilities because it has the architecture of MVC patterns and contains important concepts such as DI (dependency injection) and IoC (Inversion of control). It is also convenient because the framework itself provides swagger documentation.)
        \end{description}
    \end{enumerate}
    \item Which programming language and Why?
    \begin{enumerate}
        \item TypeScript
        \begin{figure}[htbp]
        \centerline{\includegraphics[width=3cm]{typescript.png}}
        \caption{TypeScript}
        \label{fig}
        \end{figure}
        \begin{description}
            \item TypeScript is a programming language developed and maintained by Microsoft. It is superset of JavaScript and adds optional static typing to the language. TypeScript is designed for the development of large applications and transcompiles to JavaScript. Since grammatical errors are checked at the time of compilation, it has the advantage of being able to develop more stably than script languages in which errors occur in runtime environments. Another advantage of TypeScript is that it can maximize the functionality of development tools when writing code. Visual Studio Code, which is widely used in development, is optimized for type script development because the inside of the tool is written in type script. Typically, the automatic code completion function is superior to JavaScript.
        \end{description}
        \item Java
        \begin{figure}[htbp]
        \centerline{\includegraphics[width=3cm]{Java.png}}
        \caption{Java}
        \label{fig}
        \end{figure}
        \begin{description}
            \item Java is a high-level, class-based, object-oriented programming language that is designed to have as few implementation dependencies as possible. It is a general-purpose programming language intended to let programmers write once, run anywhere meaning that compiled Java code can run on all platforms that support Java without the need for recompilation. Java applications are typically compiled to bytecode that can run on any Java virtual machine (JVM) regardless of the underlying computer architecture. The syntax of Java is similar to C and C++, but has fewer low-level facilities than either of them. The Java runtime provides dynamic capabilities (such as reflection and runtime code modification) that are typically not available in traditional compiled languages. As of 2019, Java was one of the most popular programming languages in use according to GitHub, particularly for client–server web applications, with a reported 9 million developers. Java was originally developed by James Gosling at Sun Microsystems (which has since been acquired by Oracle) and released in 1995 as a core component of Sun Microsystems’ Java platform. The original and reference implementation Java compilers, virtual machines, and class libraries were originally released by Sun under proprietary licenses. As of May 2007, in compliance with the specifications of the Java Community Process, Sun had relicensed most of its Java technologies under the GPL-2.0-only license. Oracle offers its own HotSpot Java Virtual Machine, however the official reference implementation is the OpenJDK JVM which is free open-source software and used by most developers and is the default JVM for almost all Linux distributions. As of October 2021, Java 17 is the latest version. Java 8, 11 and 17 are the current long-term support (LTS) versions. Oracle released the last zero-cost public update for the legacy version Java 8 LTS in January 2019 for commercial use, although it will otherwise still support Java 8 with public updates for personal use indefinitely. Other vendors have begun to offer zero-cost builds of OpenJDK 8 and 11 that are still receiving security and other upgrades.
        \end{description}
    \end{enumerate}
\end{itemize}

\subsection{Which software in use?}
    \begin{enumerate}
        \item AWS EC2
        \begin{figure}[htbp]
        \centerline{\includegraphics[width=3cm]{AWS EC2.png}}
        \caption{AWS EC2}
        \label{fig}
        \end{figure}
        \begin{description}
            \item We chose AWS EC2 as the deploy environment. EC2 is a cloud computing service provided by Amazon Web Services. Developers can quickly implement and distribute services without having to purchase actual servers. Linux Ubuntu 20.04 was used as the EC2 operating system.
        \end{description}
        \item AWS RDS
        \begin{figure}[htbp]
        \centerline{\includegraphics[width=3cm]{AWSRDS.png}}
        \caption{AWS RDS}
        \label{fig}
        \end{figure}
        \begin{description}
            \item RDS is a distributed relational database serviced by Amazon Web Services (AWS). It is a web service that operates in the cloud designed to simplify the setting, operation, and scaling of relational databases within an application. If DB is installed directly inside EC2, it is difficult to set up and manage. If you use RDS, you can easily access the DB only with a DB client.
        \end{description}
        \item AWS S3
        \begin{figure}[htbp]
        \centerline{\includegraphics[width=3cm]{AWSS3.png}}
        \caption{AWS S3}
        \label{fig}
        \end{figure}
        \begin{description}
            \item AWS S3 is an object storage service that provides industry-leading scalability, data availability, security and performance. AWS S3 allows users to store and protect the desired amount of data in a variety of use cases, including data rakes, websites, mobile applications, backup and restore, archives, enterprise applications, IoT devices, and big data analysis. AWS S3 provides management to optimize, structure, and organize access to data to meet specific business, organization, and compliance requirements. In our project, instead of storing the image data transmitted from the Android application on the server, we will save it in AWS S3.
        \end{description}
        \item Figma
        \begin{figure}[htbp]
        \centerline{\includegraphics[width=3cm]{figma.png}}
        \caption{Figma}
        \label{fig}
        \end{figure}
        \begin{description}
            \item Figma is a vector graphics editor and prototyping tool which is primarily web-based, with additional offline features enabled by desktop applications for macOS and Windows. The Figma Mirror companion apps for Android and iOS allow viewing Figma prototypes in real-time on mobile devices. The feature set of Figma focuses on use in user interface and user experience design, with an emphasis on real-time collaboration.
        \end{description}
        \item Github
        \begin{figure}[htbp]
        \centerline{\includegraphics[width=3cm]{github.png}}
        \caption{Github}
        \label{fig}
        \end{figure}
        \begin{description}
            \item GitHub is Microsoft’s web service that hosts source code based on distributed version control software git and supports collaboration support functions. It is currently the most popular source code hosting service and software development platform.
        \end{description}
        \item CATIA
        \begin{figure}[htbp]
        \centerline{\includegraphics[width=7cm]{CATIA.png}}
        \caption{CATIA}
        \label{fig}
        \end{figure}
        \begin{description}
            \item It is a 3D CAD and PLM software developed and sold by Dassault Systems in France. CATIA (Computer Aided Three Dimension Interactive Application) stands for interactive applied three-dimensional computer design, and is an integrated CAD/CAM/CAE SYSTEM that can handle product planning to production in batches. Although precise mathematical definitions are strong in curved modeling, which is important, on the contrary, if the user does not know the meaning of these precise mathematical definitions of these shapes, there is a problem that is quite difficult to learn. In particular, it is essential software for design in the aircraft and automobile industries due to its many curved and precise designs through Surface Modeling, and is widely used in areas that require precise design such as mold design. Considering that Dassault Systems is a company that makes mirage fighters, it is the software that originally built airplanes. It has begun to be developed for use in the design of the aviation and space industries, and now the area of use is wide.
        \end{description}
        \item NGINX
        \begin{figure}[htbp]
        \centerline{\includegraphics[width=3cm]{NGINX.png}}
        \caption{NGINX}
        \label{fig}
        \end{figure}
        \begin{description}
            \item Nginx is a lightweight web server specialized for simultaneous access processing. It is sometimes used as an HTTP Web Server that responds to static files that meet your request when you receive a request from a client, or as a load balancer that can reduce the load on the WAS server by using it as a Reverse Proxy Server.
        \end{description}
    \end{enumerate}
\begin{itemize}
    \item Cost
        \begin{table}
    \caption{Cost of Software}
    \label{t1}
    \centering
    \begin{tabular}{|p{1cm}|p{2cm}|p{4.5cm}|}
    \noalign{\smallskip}\noalign{\smallskip}\hline
    Software & Task Description & Cost \\
    \hline
    AWS EC2 &	Virtual Server &	\$0  \\
    \hline
    AWS RDS	& Remote Database &	\$0   \\
    \hline
    AWS S3 &	Image Repository &	\$0 or \$1  \\
    \hline
    Figma &	UI design tool &	\$0   \\
    \hline
    Github &	Remote repository &	\$0   \\
    \hline
    CATIA &	3D Modeling & About 23 ,000 euro / About €1,470 per year / for Education : €99 per year
    \\
    \hline
    NGINX &	Web Server &	\$0   \\
    \hline
    \end{tabular}
    \end{table}
    \end{itemize}
\subsection{Development Environment}
    \begin{enumerate}
        \item Provide clear information of development environment
        \begin{itemize}
            \item Window 10
            \begin{itemize}
                \item 2.80GHz, Core Intel i7
                \item 16GB Memory
            \end{itemize}
            \item Visual Studio Code 1.62.0
            \item TypeScript 4.3.5
            \item Node 14.17.6
        \end{itemize}
        \item Provide clear information of development environment
        \begin{itemize}
            \item Window 10
            \begin{itemize}
                \item 1.50GHz, Core Intel i5-1035G4
                \item 8GB Memory
            \end{itemize}
            \item Android Studio 4.2.1
            \item Android Emulator 30.6.5
            \item Android SDK Platform-Tools 31.0.2
            \item Compile SDK Version 30(API 30: Android 11.0(R))
            \item Intel x86 Emulator Accelerator 7.6.5
            \item Android SDK Build-Tools 32-rc1 30.0.3 (Build Tools Version)
            \item Android Gradle Plugin Version 4.2.1
            \item Gradle Version 6.7.1
            \item Junit:4.+
            \item JDK Version JAVA 16.0.2
        \end{itemize}
    \end{enumerate}


\section{Specification}
\subsection{Software}
\begin{enumerate}
    \item Loading page
    \begin{description}
        \item The intro screen was made into a closet image, and the duration of the intro screen was implemented as 3 seconds.
    \end{description}
    \item Hanger type page
    \begin{description}
        \item It is the first part of the category below. It is a page where you can manage hanger-type closets. The clothes registered in the hanger-type closet are shown in the form of images. Three icons are displayed on the upper right so that stylers, ventilation systems, and search functions can be used, respectively.
    \end{description}
    \begin{itemize}
        \item Adding/Deleting clothes
        \begin{description}
            \item Users can directly add and delete clothes registered in the closet. The registration of clothes is done in the following way.
        \end{description}
        \begin{enumerate}
            \item Top/Bottom/Others
            \begin{description}
                \item The major classification for classifying clothes was decided as the first branch point. It distinguishes whether the clothes the user wants to register are top or bottom, or classifies them so that other clothes (hats, mufflers, etc.) can be selected.
            \end{description}
            \item Top(long-sleeved/short-sleeved), Bottom (long pants/shorts)
            \begin{description}
                \item If the user chooses top, bottom, or other, it will move on to the next screen. The second branching point shows a different screen for each option. First, if you choose 'top', distinguish whether it is long-sleeved or short-sleeved. Next, if you choose 'bottom', distinguish whether it is long pants or shorts. If the length of clothes is classified as long and short as the second branch, we expect the effect of allowing users to search for clothes according to the season. Finally, if 'other' is selected, the branch point is skipped because there is no need to distinguish the length.
            \end{description}
            \item Color
            \begin{description}
                \item The third branch designates the color of the clothes. On the day of going out, it was subdivided so that users can easily and quickly search for clothes in preferred colors. When the user selects the color of the clothes, it is handed over to the next screen.
            \end{description}
            \item Material
            \begin{description}
                \item The fourth branch selects the material of the clothes. Some people determine the material of clothes according to the season and weather. Some people decide the material of clothes according to their style. Based on users' preferences, materials can be divided to quickly find clothes that users want. Materials are divided into cotton, poly, acrylic, napping, cashmere, linen, and all.
            \end{description}
            \item Choose a hanger-type number
            \begin{description}
                \item Finally, choose which hanger to hang in the closet. It is information that can help users choose one piece of clothing they want. As the most clearly distinguishable information of clothes, it is the smallest branch point and the standard that can be practically distinguished within the closet. Hanger-type numbers are given from number 1.
            \end{description}
            \item Photo Registration
            \begin{description}
                \item It is a page where you can register pictures of clothes. The information appears when the list appearing on the initial page of the hanger-type closet page is displayed. When the user presses Add Photo, it approaches the user gallery. When you select a picture, it returns to the original application. Since it is the last page, finally add a "Save Clothes Information" button that can store clothes.
            \end{description}
            \item "Save clothes information" button
            \begin{description}
                \item Press the Save Clothes Information button to output a confirmation message once again. In the confirmation message, it is divided into 'Yes' and 'No', and when selecting 'Yes', the information is sent to the DB and a pop-up message that it has been saved is output. When selecting 'No', return to the 'Photo Registration' page.
            \end{description}
        \end{enumerate}
        \item Searching for clothes
        \begin{description}
            \item It is an event generated when clicking the 'magnifier' icon at the top right of the hanger-type closet page. The list is sorted according to preferred criteria based on information on clothes registered by the user. Within one page, the criteria registered by the user are displayed on the screen once again. After selecting the information on the desired clothes, pressing the 'Search' button at the bottom to perform a search.
        \end{description}
        \item Ventilation system
        \begin{description}
            \item It is an event generated when clicking the 'wind' icon in the upper right corner of the hanger-type closet page. When the icon is clicked, a confirmation message about the use of the ventilation system is output. The answer to the message is divided into a "Yes" button and a "No" button. Press the "Yes" button to display a pop-up message saying that the ventilation system is running and return the screen to the hanger-type closet page. Pressing the "No" button returns to the hanger-type closet page without a separate screen.
        \end{description}
        \item Styler
        \begin{description}
            \item It is an event generated when clicking the 'iron' icon in the upper right corner of the hanger-type closet page. When you click the icon, the screen is turned over to a page where the Styler function can be used. On this page, the user must select the date and time, and clothes. Likewise, when the 'Start Styler Function' button is pressed, a confirmation message is output. Pressing the "Yes" button between the "Yes" button and the "No" button will display a pop-up message saying that the styler is running and return the screen to the hanger-type closet page. Pressing the "No" button returns to the hanger-type closet page without a separate screen.
        \end{description}
    \end{itemize}
    \item Drawer type page
    \begin{description}
        \item It is the second part of the category below. It is a page where you can manage a drawer-type closet. It shows a list of six drawers in a drawer-type closet. Also, each drawer makes a button that can adjust the temperature and humidity.
    \end{description}
    \begin{itemize}
        \item Temperature control
        \begin{description}
            \item Press the "temperature" button to launch a pop-up on temperature selection and create a "Yes" button and a "No" button at the bottom. If you press the "Yes" button, you will return to the drawer-type closet page with a pop-up message saying that the set temperature is applied to the drawer. Pressing the "No" button returns to the drawer-type closet page without a separate screen.
        \end{description}
        \item Humidity control
        \begin{description}
            \item Press the "humidity" button to launch a pop-up on humidity selection and create a "Yes" button and a "No" button at the bottom. When the "Yes" button is pressed, a pop-up message appears saying that the set humidity is applied to the drawer and returns to the drawer-type closet page. Pressing the "No" button returns to the drawer-type closet page without a separate screen.
        \end{description}
    \end{itemize}
    \item Calendar and Weather page
    \begin{description}
        \item It is the third part of the category below. Set the calendar to the initial plane.
    \end{description}
    \begin{itemize}
        \item Additional schedule
        \begin{description}
            \item After selecting the desired date, press the 'Register Schedule' button under the calendar. Display a text box that allows you to register the contents of the schedule as a pop-up message. In this pop-up, the "Yes" button and the "No" button are added to the bottom to add the contents of the schedule within the calendar and display a message that the schedule has been registered when selecting the "Yes" button. When selecting the 'No' button, return to the calendar/weather page without a separate screen.
        \end{description}
        \item Weather check
        \begin{description}
            \item Click the weather icon to go to the site where you can check the weather. It helps users choose clothes according to the weather.
        \end{description}
    \end{itemize}
\end{enumerate}

\subsection{Hardware}
\begin{enumerate}
    \item Hanger type space
    \begin{description}
        \item The hanger-type closet can be divided into four parts. It is divided into closet, styler, partition, and ventilator.
    \end{description}
    \begin{itemize}
        \item Closet
        \begin{description}
            \item The hanger-type closet imitated the elevator-type parking tower. I thought it was a suitable method for a hanger-type closet because the storage space was fixed and easy to manage for each item. Move the clothes along the rail to the designated place. In the closet, rails were designed and spaces were separated like branches coming from trees so that various clothes could be stored.
        \end{description}
        \item Styler
        \begin{description}
            \item While performing the styler function, it also serves as the entrance to the closet. Space efficiency was improved by unifying rails moving from the entrance to the closet. The shape of the rail is a "U" shaped rail that enters the back of the entrance and leads to the closet.
        \end{description}
        \item Partition
        \begin{description}
            \item The partition is located in the closet. It is a role that separates the space of each clothing. Like rails, it is divided into branches from large to small partitions. It serves to prevent static electricity, clothing stains, lint, and dust generation.
        \end{description}
        \item Ventilation fan
        \begin{description}
            \item The ventilator is located below the closet space. Each side is pierced in the form of a net to prevent damage to the clothes by applying wind directly to the clothes. Since the closet is somewhat away from the entrance, it serves to circulate the air in the closet. Block dust and odor generation in the closet in advance.
        \end{description}
    \end{itemize}
    \item Drawer type space
    \begin{itemize}
        \item Exterior of closet
        \begin{description}
            \item In order to take the overall design neatly, all parts except the lower entrance were blocked. The entrance is fixed at the bottom of the closet. The entrance at the bottom is also made so that there is no protrusion when viewed from the outside.
        \end{description}
        \item Interior of closet
        \begin{description}
            \item The drawer-type closet imitated the vertical circulation parking tower. I thought it was a suitable method for a drawer-type closet because I had to move and maintain a drawer that was larger than clothes. There are a total of six drawers, and each drawer is numbered. The structure of the drawer was designed in the form of a rectangular parallelepiped with width*length*height(1*2*3). Basically, six drawers are connected, and rails are placed at the entrance so that only the drawers can move in the right position.
        \end{description}
    \end{itemize}
    \item Display
    \begin{description}
        \item The display is located in the center of the entrance of the hanger-type closet. It operates in the form of a touch screen and is configured the same as functions in the application. It allows users to manage with only a closet without a cell phone.
    \end{description}
\end{enumerate}

\subsection{Software and Hardware Interaction}
\begin{enumerate}
    \item If you select an item you want to take out of the hanger-type page/drawer-type page within the application, a confirmation message is output as to whether you want to enter or not. The answer to the message is divided into a "Yes" button and a "No" button. Press the "Yes" button to display a pop-up message saying that you will enter and exit clothes/drawers. In the case of a hanger-type closet, the clothes are moved along the rail from the closet to the entrance. In the case of a drawer-type closet, the drawer is circulated to the entrance. Once the item has arrived at the entrance, it shows the clothes to the user through the entrance. If you press the 'No' button, no event will occur.
    \item If you use the Styler function at the top right of the hanger-type closet page, you can move your clothes from the closet to the entrance in advance considering the date and time before the styler operates. The styler function is terminated at the time the user wants and clothes are shown to the user through the entrance.
    \item When the ventilation system at the top right of the hanger-type closet page is implemented, the ventilation machine at the bottom of the hanger-type closet begins to rotate. When the air quality is checked and it is determined that sufficient ventilation has been made in the closet, the ventilator automatically stops running.
    \item If you set the status of the drawer through the Temperature button and Humidity button in each drawer on the drawer-type closet page, the drawer in the actual closet changes the environment inside the drawer according to the user's preference.
    \item When the closet is managed through the application and the settings are changed, the display in the closet is also updated. Conversely, even if it is changed through a display in the closet, the application is also updated in the same manner.
\end{enumerate}
\end{document}
